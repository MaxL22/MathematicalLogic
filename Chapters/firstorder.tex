\chapter{First-Order Logic}

Introductory example: 
\begin{itemize}
	\item If it rains ($p$) I take the umbrella ($q$) 
	\item I do not take the umbrella  ($\neg q$)
	\\ \bline
	\item Therefore it does not rain
\end{itemize}

This lies in \textbf{propositional logic}:
$$ \{p \rightarrow q, \neg q\} \models \neg p $$
We can transform it into UNSAT and by using DPP we can see that it outputs $\square$, so the thing is UNSAT, so we confirmed that our example is correct.\\

Another, different, example:
\begin{itemize}
	\item Every man is mortal ($p$)
	\item Socrates is a man ($q$)
	\\ \bline
	\item Therefore Socrates is a mortal ($r$)
\end{itemize}
They are all atomic, we \textbf{can't relate them using propositional logic}, trying DPP on this is just dumb, starting from $\{p,q\} \models r$, obviously it goes to $\emptyset$. We \textit{need a more expressive language} able to express these kind of concepts.\\
To handle reasoning such as the syllogism about Socrates we must adopt richer (more expressive) language.\\

\paragraph{Syntax:} We'll enrich syntax by adding \textbf{new} kinds of \textbf{symbols}: 
\begin{itemize}
	\item Quantifiers
	\item Predicate symbols
	\item Function symbols (including constants)
	\item Variables
\end{itemize}

\paragraph{Semantics:} We attach \textbf{new meanings} to new concepts, to assign a formal way to interpret the new "ingredients" of the language, with the aim of being able to express more refined aspects of the possible "worlds" that we want to model.\\

\newpage

Try to develop \textbf{calculi fit to be automated} , to (semi) decide problems like the earlier syllogism, which can be \textbf{formalized} as:
\begin{itemize}
	\item $\forall x (U(x) \rightarrow M(x))$
	\item $U(s)$
	 \\ \bline
	\item $M(s)$
\end{itemize}
For every element $x$ of the group $U$ (men), the element is also in group $M$ (mortal). So if $s$ (Socrates) is in $U$, then $s$ is in $M$.\\

So 
$$ \{\forall x (U(x) \rightarrow M(x)), U(s)\} \models^? M(s) $$
But it will make sense after introducing syntax and semantics.\\

Preprocessing, going in Normal Form, getting first-order clauses and another UNSAT problem
$$ \{ \{\neg U(x), M(x)\}, \{U(s)\}, \{\neg M(s)\} \} \;\;\;\; UNSAT $$

Herbrand's theory: we may substitute $x$ with $s$ 
$$ \{ \{\neg U(s), M(s)\}, \{U(s)\}, \{\neg M(s)\} \} \;\;\;\; UNSAT $$
We can all consider these propositional letters:
$$ \{\{\neg p, q\}, \{p\}, \{\neg q\}\} \;\;\;\; PROP \; UNSAT $$
Then we can just use resolution, showing that it's refutable and thus proving true the first statement.
$$
\AxiomC{$p$}
\AxiomC{$\neg p q$}
\BinaryInfC{$q$}
\AxiomC{$\neg q$}
\BinaryInfC{$\square$}
\DisplayProof
$$

\newpage

Adding \textbf{another layer}:
\begin{itemize}
	\item Every man is mortal 
	\item The father of every man is a man
	\item Socrates is a man
	\\ \bline
	\item The father of Socrates is mortal
\end{itemize}

This can be \textbf{formalized} into
\begin{itemize}
	\item $\forall x (U(x) \rightarrow M(x))$
	\item $\forall x (U(x) \rightarrow U(f(x)))$
	\item $U(s)$
	\\ \bline
	\item $M(f(s))$
\end{itemize}
Where the function $f$ represents a \textbf{relation} (being father) \textbf{to an element}, Socrates $s$ in this case.\\

$$ 
\{ \{\neg U(x), M(x)\}, \{\neg U(x), U(f(x))\}, \{U(s)\}, \{\neg M(f(s))\}\} \;\;\;\; UNSAT $$
Using Herbrand's theory: we can use $s$ and $f(s)$, but also $f(f(s))$, $f(f(f(s)))$, \dots, there are \textbf{infinitely many different substitutions}. We need to find the "\textit{clever}" one, not just any.\\

In this case $x$ becomes $f(s)$
$$
\AxiomC{$\neg U(f(s)), M(f(s))$}
\AxiomC{$\neg M(f(s))$}
\BinaryInfC{$\neg U(f(s))$}
\AxiomC{$\neg U(s), U(f(s))$}
\BinaryInfC{$\neg U(s)$}
\AxiomC{$U(s)$}
\BinaryInfC{$\square$}
\DisplayProof
$$
Which means UNSAT.\\

% End L18

\newpage

\section{Syntax of First Order Logic}

At propositional level, we considered a unique language (the set of propositional letters $\mathcal{L}$). From this we derived the set $\fl$ of propositional formulas.\\

In First order logic we'll consider \textbf{several distinct languages} ("elementary languages", in the sense that it has to do with elements).\\

\begin{definition}
	An \textbf{elementary language} $\mathcal{L} = (\mathcal{P}, \mathcal{F}, \alpha, \beta)$, where: 
	\begin{itemize}
		\item $\mathcal{P}$ is the set of all "\textbf{predicates}" ($\mathcal{P} \neq \emptyset$)
		\item $\mathcal{F}$ is the set of symbols, called "\textbf{function symbols}" ($\mathcal{P} \cap \mathcal{F} = \emptyset$)
		\item arity functions: 
		$$
		\begin{cases}
			\alpha: \mathcal{P} & \rightarrow \mathbb{N} \\
			\beta: \mathcal{F} & \rightarrow \mathbb{N}\\
		\end{cases}
		$$
		$\alpha$ assigns the "\textbf{arity}" of \textbf{each} $p \in \mathcal{P}$, $\beta$ assigns the "\textbf{arity}" of \textbf{each} $f \in \mathcal{F}$.\\
	\end{itemize}
\end{definition}
	
"\textbf{Ingredients}" common to \textbf{every elementary language}:
\begin{itemize}
	\item $\mathcal{V}$: an \textbf{infinite set of symbols} ($\mathcal{V} \cap \mathcal{P} \neq \emptyset$, $\mathcal{V} \cap \mathcal{F} \neq \emptyset$), called "\textbf{individual variables}"
	\item \textbf{Connectives:} $\wedge, \vee, \neg, \rightarrow$ (and all derived connectives $\bot, \top, \leftrightarrow, \, \dots$)
	\item \textbf{Quantifiers:} $\forall, \exists$
\end{itemize}

The formulas of the formal language $\fl$ (formulas over the elementary language $\mathcal{L}$) will be defined from $\mathcal{L}: (\mathcal{P}, \mathcal{F}, \alpha, \beta)$ together with the common "ingredients".\\
	
\paragraph{Key stipulation:} If "$=$" $\in \mathcal{P}$, then $\alpha(=) = 2$ (its \textbf{arity is always 2}), and the language with $= \in \mathcal{P}$ is called a \textbf{language with identity}.\\

\subsection{Formulas in $\fl$}
The definition of formulas of $\fl$ ($F \in \fl$) is given \textbf{by layers} and \textbf{inductively} on each one of the two layers.\\

\paragraph{First layer:} Terms of $\mathcal{L}$ (also called $\mathcal{L}$-terms).\\

\begin{definition}
	Let $\mathcal{L}: (\mathcal{P}, \mathcal{F}, \alpha, \beta)$. Then the \textbf{set $\mathcal{T}_\mathcal{L}$ of $\mathcal{L}$-terms} is defined as follows:
	\begin{itemize}
		\item \textbf{Base:}
		\begin{itemize}
			\item Every $x \in \mathcal{V}$ ($x$ is an \textbf{individual variable}, is an $\mathcal{L}$-term, $x \in \mathcal{T}_\mathcal{L}$)
			\item Every $C \in \mathcal{F}$, such that $\beta(c) =0$ is an $\mathcal{L}$-term ($c \in \mathcal{T}_\mathcal{L}$, it's a \textbf{constant symbol})
		\end{itemize} 
		\item \textbf{Step:} If $f \in \mathcal{F}$ is such that $\beta(f) = n > 0$ and $t_1, t_2, \, \dots \, , t_n \in \mathcal{T}_\mathcal{L}$ then $f(t_1, t_2, \, \dots \, , t_n)$ is an $\mathcal{L}$-term ($\in \tl$)
	\end{itemize} 
	Nothing else is an $\mathcal{L}$-term.\\
\end{definition}

\newpage

\paragraph{Second Layer:} $\mathcal{L}$-formulas.\\

\begin{definition}
	Let $\mathcal{L}: (\mathcal{P}, \mathcal{F}, \alpha, \beta)$. The \textbf{set $\fl$ of $\mathcal{L}$-formulas} is defined as follows: 
	\begin{itemize}
		\item \textbf{Base:} if $P \in \mathcal{P}$ and $\alpha(P) = n$, and $t_1, t_2, \, \dots \, , t_n \in tn$ then $P(t_1, t_2, \, \dots \, , t_n)$ is an $\mathcal{L}$-formula called Atomic (cannot split an atomic formula into another formula, only into terms)
		\item Step:
		\begin{itemize}
			\item If $A,B \in \fl$ then $(A \wedge B), (A \vee B), (\neg A), (A \rightarrow B)$ are $\mathcal{L}$-formulas.
			\item if $A \in \fl$ and $x \in \mathcal{V}$ then $(\forall x A)$ and $(\exists x A)$ are $\mathcal{L}$-formulas $(\in \fl)$
		\end{itemize} 
	\end{itemize}
	Nothing else is an $\mathcal{L}$-formula.\\
\end{definition}

Exercise: Provide a notion of "certificate" for $\mathcal{L}$-terms and $\mathcal{L}$-formulas analogous to the notion of $\mathcal{L}$-constructions for propositional logic.\\

\begin{remark}
	For simplicity's sake, we shall \textbf{omit parentheses} when there is \textbf{no risk of ambiguity}.\\
\end{remark}

\begin{terminology}
	An $\mathcal{L}$-term is called "\textbf{ground}" if it is \textbf{built without using variables}.\\
\end{terminology}

\begin{terminology}
	An occurrence of a variable $x \in \mathcal{V}$ in a formula $A \in \fl$ is "\textbf{bound}" ("constrained") if it occurs in a subformula $B \preceq A$ of the form $(\forall x C)$ or $(\exists x C)$ ($x$ is in the scope of some quantifier, i.e., it "can be seen" from a $\forall$ or $\exists$).
\end{terminology}

\begin{terminology}
	If an occurrence of $x \in \mathcal{V}$ in $A \in \fl$ is not bound then $x$ \textbf{occurs free} in $A$.\\
	A \textbf{variable} $x \in \mathcal{V}$ occurs free in $A \in fl$ if some of its occurrences are free (otherwise the variable is bound).\\
	We denote with $FV(A)$ the \textbf{set of variables occurring free} in $A$.\\
	%Add example, from "Notice".\\
\end{terminology}

\begin{definition}
	An \textbf{$\mathcal{L}$-sentence} or \textbf{closed $\mathcal{L}$-formula} is an $\mathcal{L}$-formula \textbf{without free variables} ($FV(A) = \emptyset$).\\
\end{definition}

With the notation: $A [\sfrac{t}{x}]$ where $A \in \fl$, $x \in \mathcal{V}$, $t \in \tl$ we mean the $\mathcal{L}$-formula obtained by \textbf{replacing} simultaneously all and only the \textbf{free occurrences} of $x$ with $t$.\\

\newpage

\section{Semantics of First order logic}

To provide an \textbf{interpretation of an $\mathcal{L}$-sentence} (whether it's true or false) in every given "circumstance" (or "possible world") we have to: 
\begin{enumerate}
	\item Fix and formalize the \textbf{notion of} "\textbf{circumstance}" (or "possible world", in the propositional case, a "possible world" is formalized just as a truth value assignment)
	\item Formalize the \textbf{interpretation of $\mathcal{L}$-sentences} (and (ground) $\mathcal{L}$-terms) in every formal "possible world".
\end{enumerate}

\paragraph{Idea (Tarskian semantics):} The \textbf{truth} is the \textbf{correspondence with the state of things}, e.g., the sentence "The snow is white" is true (in a given world) iff the actual snow is actually white.\\
We have a universe of discourse (our world) in which there are some elements (snow) with a certain property (is white). In another universe there can be an element with the same name without the same properties.\\

\subsection{$\mathcal{L}$-Structures}
Let $\lang$. An \textbf{$\mathcal{L}$-structure} is a \textbf{pair} $\mathcal{A} = (A, I)$, where $A$ is just a set, called "\textbf{universe of discourse}" (or "\textbf{domain}"). It is \textbf{required} $A \neq \emptyset$. And $I$ is a function called "\textbf{Interpretation}", defined as follows: 
\begin{itemize}
	\item for all $P \in \mathcal{P}$, $\alpha (P) = n$
	$$ I(P) \subseteq A^n $$
	$I(P)$ is an $n$-ary relation, that is, it's a set of $n$-tuples of elements of $A$.\\
	
	\item For all $f \in \mathcal{F}$, $\beta(f) = n$
	$$ I(f):  A^n \rightarrow A $$
	$I(f)$ is a function from the set of all $n$-tuples of elements of $A$ to $A$.\\
\end{itemize}

\begin{remark}
	If $c \in \mathcal{F}$, $\beta (c) = 0$, if $c$ is a constant, then 
	$$ I(c) : A^0 \rightarrow A$$
	but $A^0 = \{f: \emptyset \rightarrow A \} = \{\emptyset\}$.\\
	It takes the set containing only the $\emptyset$ and fixes an element of $A$ to it. \\
	
	The "left side" ($\emptyset$ part) carries no information, so we identify $f:A^0 \rightarrow A$ with the element $f(\emptyset)$, $I(c) \in A$.\\
\end{remark}

\begin{remark}
	If $P \in \mathcal{P}$, $\alpha (P) = 0$
	$$ I(P) \subset A^0 \implies I(P) \subseteq \{\emptyset\}$$
	So $I(P)$ can be $\emptyset$ or $A^0 = \{\emptyset\}$ (we can call them 0 and 1 respectively).\\
	
	To each relation we can associate its characteristic function $\chi$
	$$ \chi_{I(Q)}$$
	%TODO I have no idea what it should be
\end{remark}

\begin{remark}
	Let $\lang$ be a language with identity, $"=" \in \mathcal{P}$, $\alpha(=) = 2$. The semantics of "$=$" is fixed. \\
	Let $\mathcal{A} = (A,I)$ be an $\mathcal{L}$-structure, then 
	$$ I (=) \subseteq A^2 \;\;\;\;\; I (=) := \{ (a,a): a \in A\} $$
	Each element is in relation with only itself, this is the meaning of "$=$".\\
	We shall see later that $(I(a), I(b)) \in I(=)$ iff $I(a) = I(b)$ (assuming $a,b \in \tl$).\\
\end{remark}

% TODO, Idk what it means
Define the \textbf{semantics} of \textbf{$\mathcal{L}$-sentences} (and \textbf{ground $\mathcal{L}$-terms}) in \textbf{every $\mathcal{L}$-structure}. We proceed by layers
\begin{itemize}
	\item layer for (ground) $\mathcal{L}$-terms
	\item layer of $\mathcal{L}$-sentences ($\mathcal{L}$-formulas)
\end{itemize}
and inductively on each of these two layers.\\

We have to capture the Tarskian "idea". We'll need to say when an $\mathcal{L}$-structure such as $\lstruc$ "\textbf{models}" a certain property, e.g., 
$$ \mathcal{A} \models W(s)$$

% End L19

\paragraph{Key observation about the variables:} If we want to give \textbf{semantics to $\mathcal{L}$-sentences} proceeding inductively (and we're obliged to) we have a problem with quantifiers and variables
$$ \forall x \;\; Cube(x) \;\;\;\;\; \exists x \;\; Cube(x) $$

The semantics we want to define for these two examples, if we are to proceed inductively, should be based on the components of those two sentences, it should be based on the semantics of 
$$ Cube(x) $$
and there is no way to definitely assign a truth value (in any given $\mathcal{L}$-structure; it's not a sentence, it doesn't make sense) to $Cube(x)$.\\

Possible \textbf{solutions}: 
\begin{enumerate}
	\item \textbf{Enrich the interpretation}: interpret any variable $x$, too
	\item try to \textbf{get rid of variables} %What
\end{enumerate}

These two methods give the same semantics. We'll take a look at the second approach first.\\

\newpage

\subsection{Interpretation}

We are going to establish the \textbf{interpretation of quantifiers}.\\

We want, for instance, to establish whether in a given $\mathcal{L}$-structure $\mathcal{A} = (A,I)$
$$ \mathcal{A} \models \exists x F$$
The structure "models"/"renders true" the sentence on the right. We need to define this notation.\\

This is true \textbf{iff} it exists at least one element $a \in A$ such that $\mathcal{A} \models F[\sfrac{a}{x}]$.\\
While
$$ \mathcal{A} \models \forall x F $$

is true \textbf{iff} for all $a \in A$ it holds that $\mathcal{A} \models F[\sfrac{a}{x}]$.\\

We're \textbf{substituting} a variable ($a$) with an element of the universe ($x$). These expressions make no sense: we cannot substitute variables with objects of the universe. We can overcome this difficulty if every object in the universe has a name (that is, there is a term whose interpretation is that object).\\

In general, there is no need to have a name for each object of the universe we want to talk about. In many practical cases, it's usually not the case. For instance, if we want to deal in first order logic with the real numbers we want to do it with an elementary language $\mathcal{L}$ containing just finitely many symbols and able to name just enumerably many reals.\\ 

So, in order to give the semantics to $\mathcal{L}$-sentences over universes "too large" to have names for all objects in the universe, we "cheat" expanding the language $\mathcal{L}$ to give names to all objects. This expansion of the language will only be "instrumental" to give the semantics.\\

\textbf{Side effect:} variables will be mere "placeholders", they will be just syntactic devices, without any direct semantics.\\

\textbf{Expansion} of $\mathcal{L}$ to $\mathcal{L}_\mathcal{A}$:
\begin{enumerate}
	\item Let $\lang$.\\
	
	\item Let $\mathcal{A} = (A,I)$ be an $\mathcal{L}$-structure.\\
	
	\item Define the expansion $\la$ of $\mathcal{L}$ with respect to $\mathcal{A}$. $\mathcal{L}_\mathcal{A}$ is \textbf{defined} as follows: 
	\begin{enumerate}
		\item For all $a \in A$ add to the set of constants .. a new symbol $\overline a$ (the "name" of $a$)
		\item we shall interpret each $\overline a$ into $a$
	\end{enumerate} \nn
\end{enumerate}

\newpage

%TODO Wtf is this, why are there random steps
Steps for something, idk:
\begin{enumerate}
	\item We define
	$$ \mathcal{L}_\mathcal{A} := (\mathcal{P}, \mathcal{F}^\mathcal{A}, \alpha, \beta^\mathcal{A}) $$
	$$ \mathcal{F}^\mathcal{A} := \mathcal{F} \cup \{\overline a: a \in A\} \;\;\;\; (\overline a \notin \mathcal{F}) $$
	$$ \beta^\mathcal{A} = \beta \; \text{ for all } \;  f \in \mathcal{F} $$
	$$ \beta^\mathcal{A}(\overline a) = 0 \; \text{ for all }\; a \in A $$
	
	\item $\overline{\mathcal{A}} = (A, \overline I)$, $\overline{\mathcal{A}}$ is an $\la$-structure. $\overline I$ coincides with $I$ on $\mathcal{L}$, while $\overline{I} (\overline a) := a$, for all $a \in A$
\end{enumerate}

First there is $\lang$, then there are infinitely many $\mathcal{L}$-structures
$$
\begin{cases}
	\lstruc & \rightsquigarrow \la \rightsquigarrow \overline{\mathcal{A}} = (A, \overline I) \;\;\; \la \text{-structure} \\
	\mathcal{B} = (B, I) & \rightsquigarrow \mathcal{L}_\mathcal{B} \rightsquigarrow \overline{\mathcal{B}} = (B, \overline{I}) \; \; \; \mathcal{L}_\mathcal{B} \text{-structure} \\
	\vdots & \\
\end{cases}
$$

%TODO, maybe idk
%For each one we get a different expansion:
%Complete

\subsubsection{Interpretation of $\la$-ground terms}

Inductively:
\begin{itemize}
	\item \textbf{Base:} 
	\begin{itemize}
		\item old constants $\rightarrow c in \mathcal{F}, \beta(c) = 0 \rightarrow \overline I (c) := I(c)$ 
		\item new constants $\rightarrow \overline a \in \mathcal{F}^\mathcal{A} \setminus \mathcal{F} \rightarrow \overline I (\overline a) := a$ $(a \in A)$
	\end{itemize} \nn
	
	\item \textbf{Step:} Let $F \in \mathcal{F}$, $\beta (F) = n > 0$ and let $t_1, \, \dots \, , t_n$ be $\la$-terms, that are ground.\\
	Then 
	$$\overline I (F(t_1, \, \dots \, , t_n)) := (I(F))(\overline I (t_1), \, \dots \, , \overline I (t_n)) $$
	So this 
	$$ I(F): A^n \rightarrow A $$
\end{itemize}

By induction, for each ground $\la$-term, the interpretation of $t$ is guaranteed to be an element of $A$, $\overline I (t) \in A$.\\

\newpage

\subsubsection{Interpretation of $\la$-sentences}

Inductively we define $\overline{\mathcal{A}} \models F$ ($\overline{\mathcal{A}}$ "modes"/"renders true" $F$) for each $F \in \la$-sentence.\\

At the end of the definition each sentence is either \textit{true} (x)or \textit{false} in $\overline{\mathcal{A}}$: 
\begin{itemize}
	\item Base; ground atomic $\mathcal{L}_\mathcal{A}$ formulas (an atomic formula is a sentence iff it is ground)
	$$ F = P (t_1, \, \dots \, , t_n)  \; \text{ for } \; P \in \mathcal{P}, \alpha(P) = n $$
	$t_1, \, \dots \, , t_n$ are $\la$-ground terms.\\
	Then we can define
	$$
	\overline{\mathcal{A}} \models P(t_1, \, \dots \, , t_n) \; \text{ iff } \; (\overline I (t_1), \, \dots \, , \overline I (t_n)) \in I (P)
	$$
	
	\item Step: 
	$$ 
	\begin{array}{c c c}
		\overline{\mathcal{A}} \models F_1 \wedge F_2 & \; \text{ iff } \; & \overline{\mathcal{A}} \models F_1 \; \text{ and } \overline{\mathcal{A}} \models F_2 \\
		\overline{\mathcal{\mathcal{A}}} \models F_1 \vee F_2 & \; \text{ iff } \; & \overline{\mathcal{A}} \models F_1 \; \text{ or } \; \overline{\mathcal{A}} \models F_2 \\
		\overline{\mathcal{A}} \models \neg F_1 & \; \text{ iff } \; & \overline{\mathcal{A}} \not \models F_1 \\
		\overline{\mathcal{A}} \models F_1 \rightarrow F_2 & \; \text{ iff } \; & \overline{\mathcal{A}} \not \models F_1 \; \text{ or } \; \overline{\mathcal{A}} \models F_2 
	\end{array}
	$$
	
	\item Step (again): 
	\begin{itemize}[label=]
		\item $\overline{\mathcal{A}} \models \forall x F$ iff for all $a \in A$, $\overline{\mathcal{A}} \models F [\sfrac{\overline a}{x}]$
		\item $\overline{\mathcal{A}} \models \exists x F$ iff there is $a \in A$ such that $\overline{\mathcal{A}} \models F [\sfrac{\overline a}{x}]$
	\end{itemize}
	\nn
\end{itemize}

By induction, for every $\la$-sentence $F$ the \textbf{truth value of} $F$ in the structure $\overline{\mathcal{A}}$ is \textbf{completely determined}, either $\overline{\mathcal{A}} \models F$ (x)or $\overline{\mathcal{A}} \not \models F$.\\

We have completed the inductive definition of the \textbf{relation} $\overline{\mathcal{A}} \models F$ for each $\mathcal{L}$-structure $\mathcal{A}$ and $\la$-sentence $F$,\\

For all $F \in \mathcal{L}$-sentence, we have that $\mathcal{A} \models F$ (as an $\mathcal{L}$-structure) iff $\overline{\mathcal{A}} \models F$ (as an $\la$-structure).\\

\begin{remark}
	$F \in \fl$ but $F$ is not in $\mathcal{L}$-sentence. Assume
	$$ FV(F) \neq \emptyset \;\;\;\;\; FV(F) \subseteq \{x_1, x_2, \, \dots \, , x_n\} $$
	that is, $F$ has some free variables.\\
	
	How can we define the notion
	$$ \mathcal{A} \models^? F $$
	(in case we are interested). If $F$ is a sentence we already specified, but if there are free variables 
	$$ \forall [F] := \forall x_1 \forall x_2 \, \dots \, \forall x_n F$$
	we call this "universal closure of $F$", $\implies \forall [F]$ is an $\mathcal{L}$-sentence (so $FV(\forall[F]) \neq \emptyset$). Universal closure of a formula.\\
	By definition: $\mathcal A \models F$ iff $\mathcal A \models \forall [F]$.\\
\end{remark}

\begin{remark}
	If $\mathcal{L}$ is a \textbf{language with identity} $\lang$, $= \in \mathcal{P}$, $\alpha(=) = 2$, and in every $\mathcal{L}$-structure (with identity) $\mathcal{A} = (A,I)$ we have the same interpretation of $=$, 
	$$ I(=) = \{(a,a): a \in A\} \subseteq A^2 $$
	
	Then $t_1, t_2$ ground $\mathcal{L}$-terms ($\la$-terms) then we have that 
	$$ \mathcal{A} \models t_1 = t_2 \;\;\;\; (\overline{\mathcal{A}} \models t_1 = t_2) $$
	iff
	$$ I(t_1) = I(t_2) \;\;\;\; (\overline I (t_1) = \overline I (t_2)) $$
\end{remark}


\paragraph{Alternative (but equivalent) definition of $\mathcal{A} \models F$:} We'll say $\mathcal{A} \models' F$ to distinguish them, but they're functionally the same.\\

Here, the variables will play a major role (instead of being just placeholders). The two definitions $\models$, $\models'$ are completely equivalent $\models = \models'$ ($\mathcal{A} \models F$ iff $\mathcal{A} \models' F$).\\

\begin{definition}
	An (interpretational) \textbf{environment} in an $\mathcal{L}$-structure $\mathcal{A} = (A,I)$ is just a function $v: \mathcal{V} \rightarrow A$ (map every variable to an element of the universe).\\
\end{definition}

\begin{definition}
	\textbf{Variant of an environment}, having $v: \mathcal{V} \rightarrow A$, for a variable $x \in \mathcal{V}$ and an element of the universe $a \in A$
	$$ v[\sfrac{a}{x}] (y) = 
	\begin{cases}
		v(y) & \text{ if } y \neq x \\
		a & \text{ if } y = x
	\end{cases}
	$$
\end{definition}
$v[\sfrac{a}{x}]$ is the variant of $v$ with respect to $x,a$. Clearly it's an environment.\\

Interpretation of $\mathcal{L}$-terms (also non-ground terms):
\begin{itemize}
	\item For all $x \in \mathcal{V}$ and each environment $v: \mathcal{V} \rightarrow A$
	$$ I_v (x) := v(x) $$
	\item while $I_v (F(t_1, \, \dots \, , t_n)) := (I(F))(T_v(t_1), \, \dots \, , I_v (t_n))$
\end{itemize} \nn

\textbf{Define:} $\mathcal{A} \models_v F$, it's the \textbf{interpretation with regard to some environment} $v$, "$\mathcal{A}$ models $\mathcal{F}$ in the environment $v$".
\begin{itemize}
	\item $\mathcal{A} \models_v P(t_1, \, \dots \, , t_n)$ iff $(I_v (t_1), \, \dots \, , I_v (t_n)) \in I(P)$
	$$
	\begin{cases}
		\mathcal{A} \models_v F_1 \wedge F_2 \; \text{ iff } \; \mathcal{A} \models_v F_1 \; \text{ and } \; \mathcal{A} \models_v F_2 \\
		\vdots 
	\end{cases}
	$$
	\item $\mathcal{A} \models_v \forall x \in F$ iff for all $a \in A$ $\mathcal{A} \models_{v [\frac{a}{x}]} F$.\\
	\item $\mathcal{A} \models_v \exists x \in F$ iff for all $a \in A$ $\mathcal{A} \models_{v [\frac{a}{x}]} F$
\end{itemize}

\newpage

But we want to define the notion without dependence on the environment
$$ \mathcal{A} \models' F$$
iff for all $v: \mathcal{V} \rightarrow A$: $\mathcal{A} \models_v F$ .\\

%Comparison between definitions:
%\begin{itemize}
%	\item First definition: 
%	$$ 
%	
%	$$
%	
%	\item Second definition:
%\end{itemize}

% End L20 