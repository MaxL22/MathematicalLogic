\chapter{First-Order Logic}

Introductory example: 
\begin{itemize}
	\item If it rains ($p$) I take the umbrella ($q$) 
	\item I do not take the umbrella  ($\neg q$)
	\\ \bline
	\item Therefore it does not rain
\end{itemize}

This lies in \textbf{propositional logic}:
$$ \{p \rightarrow q, \neg q\} \models \neg p $$
We can transform it into UNSAT and by using DPP we can see that it outputs $\square$, so the thing is UNSAT, so we confirmed that our example is correct.\\

Another, different, example:
\begin{itemize}
	\item Every man is mortal ($p$)
	\item Socrates is a man ($q$)
	\\ \bline
	\item Therefore Socrates is a mortal ($r$)
\end{itemize}
They are all atomic, we \textbf{can't relate them using propositional logic}, trying DPP on this is just dumb, starting from $\{p,q\} \models r$, obviously it goes to $\emptyset$. We \textit{need a more expressive language} able to express these kind of concepts.\\
To handle reasoning such as the syllogism about Socrates we must adopt richer (more expressive) language.\\

\paragraph{Syntax:} We'll enrich syntax by adding \textbf{new} kinds of \textbf{symbols}: 
\begin{itemize}
	\item Quantifiers
	\item Predicate symbols
	\item Function symbols (including constants)
	\item Variables
\end{itemize}

\paragraph{Semantics:} We attach \textbf{new meanings} to new concepts, to assign a formal way to interpret the new "ingredients" of the language, with the aim of being able to express more refined aspects of the possible "worlds" that we want to model.\\

\newpage

Try to develop \textbf{calculi fit to be automated} , to (semi) decide problems like the earlier syllogism, which can be \textbf{formalized} as:
\begin{itemize}
	\item $\forall x (U(x) \rightarrow M(x))$
	\item $U(s)$
	 \\ \bline
	\item $M(s)$
\end{itemize}
For every element $x$ of the group $U$ (men), the element is also in group $M$ (mortal). So if $s$ (Socrates) is in $U$, then $s$ is in $M$.\\

So 
$$ \{\forall x (U(x) \rightarrow M(x)), U(s)\} \models^? M(s) $$
But it will make sense after introducing syntax and semantics.\\

Preprocessing, going in Normal Form, getting first-order clauses and another UNSAT problem
$$ \{ \{\neg U(x), M(x)\}, \{U(s)\}, \{\neg M(s)\} \} \;\;\;\; UNSAT $$

Herbrand's theory: we may substitute $x$ with $s$ 
$$ \{ \{\neg U(s), M(s)\}, \{U(s)\}, \{\neg M(s)\} \} \;\;\;\; UNSAT $$
We can all consider these propositional letters:
$$ \{\{\neg p, q\}, \{p\}, \{\neg q\}\} \;\;\;\; PROP \; UNSAT $$
Then we can just use resolution, showing that it's refutable and thus proving true the first statement.
$$
\AxiomC{$p$}
\AxiomC{$\neg p q$}
\BinaryInfC{$q$}
\AxiomC{$\neg q$}
\BinaryInfC{$\square$}
\DisplayProof
$$

\newpage

Adding \textbf{another layer}:
\begin{itemize}
	\item Every man is mortal 
	\item The father of every man is a man
	\item Socrates is a man
	\\ \bline
	\item The father of Socrates is mortal
\end{itemize}

This can be \textbf{formalized} into
\begin{itemize}
	\item $\forall x (U(x) \rightarrow M(x))$
	\item $\forall x (U(x) \rightarrow U(f(x)))$
	\item $U(s)$
	\\ \bline
	\item $M(f(s))$
\end{itemize}
Where the function $f$ represents a \textbf{relation} (being father) \textbf{to an element}, Socrates $s$ in this case.\\

$$ 
\{ \{\neg U(x), M(x)\}, \{\neg U(x), U(f(x))\}, \{U(s)\}, \{\neg M(f(s))\}\} \;\;\;\; UNSAT $$
Using Herbrand's theory: we can use $s$ and $f(s)$, but also $f(f(s))$, $f(f(f(s)))$, \dots, there are \textbf{infinitely many different substitutions}. We need to find the "\textit{clever}" one, not just any.\\

In this case $x$ becomes $f(s)$
$$
\AxiomC{$\neg U(f(s)), M(f(s))$}
\AxiomC{$\neg M(f(s))$}
\BinaryInfC{$\neg U(f(s))$}
\AxiomC{$\neg U(s), U(f(s))$}
\BinaryInfC{$\neg U(s)$}
\AxiomC{$U(s)$}
\BinaryInfC{$\square$}
\DisplayProof
$$
Which means UNSAT.\\

% End L18